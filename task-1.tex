\documentclass[a4paper,12pt]{scrartcl}
%test
%%%%%%%%%%%%%%%%%%%%%%%%%%%%%%%%%%%%%%%%%
\def\Nr{1}					%Hier die Nummer des Übungsblatts eintragen
\def\NameA{Dario Estermann}	%Hier den Name eintragen
\def\MatA{10054384}			%Hier die Matrikelnummer eintragen
\def\NameB{xxxxxxx}		%Hier den Name des 2. Gruppenmitglieds eintragen
\def\MatB{xxxxx}			%Hier die Matrikelnummer des 2. Gruppenmitglieds eintragen
\def\NameC{}				%Name 3. Gruppenmitglied
\def\MatC{}					%Matrikelnummer 3. Gruppenmitglied
\def\NameD{}				%Name 4. Gruppenmitglied
\def\MatD{}					%Matrikelnummer 4. Gruppenmitglied
%%%%%%%%%%%%%%%%%%%%%%%%%%%%%%%%%%%%%%%%%
\usepackage[utf8]{inputenc}
\usepackage[ngerman]{babel}
\usepackage{hyperref}
\usepackage{booktabs}
\usepackage{geometry}
\usepackage{amssymb}
\usepackage{amsmath}
\usepackage{ifthen}
\usepackage{enumerate}
\usepackage{verbatim}
\usepackage{multicol}
\usepackage{algorithm}
\usepackage{xcolor}
\setlength{\marginparwidth}{2.5cm}
\usepackage{todonotes}
\usepackage{tikz}
\usepackage{forest}
\usetikzlibrary{
	arrows,
	arrows.meta,
	automata,
	calc,
	chains,
	trees,
	positioning,
	scopes,
	decorations.pathmorphing,
	shapes,
	backgrounds,
	chains,
	}

\geometry{margin=3cm, top=2.7cm}
\renewcommand{\thesection}{\arabic{section}{.}}

\begin{document}
\begin{center}
	\sffamily
	\bfseries
	\LARGE
	Datenstrukturen und Algorithmen\\
	\Large
	\vspace{.2\parskip}
	Hausübung \Nr\\
	\normalsize\normalfont
	WiSe 23/24
	\vspace{.2\parskip}
\end{center}

\begin{tabular}[t]{p{4.5cm} p{3cm}}
	\toprule
	Name & Matrikelnummer\\
	\midrule
	\NameA & \MatA\\
	\NameB & \MatB\\
	\ifx\NameC\empty\else\NameC & \MatC\\\fi
	\ifx\NameD\empty\else\NameD & \MatD\\\fi
	\bottomrule
\end{tabular}
\hfill
\begin{tabular}[t]{ccccc}
	\toprule
	A1 & A2 & A3 & Bonus & $\Sigma$ \\
	\midrule
	\\
	\bottomrule	
\end{tabular}
\hfill\\


%%%%%%%%%%%%%%%%%%%%%%%%%%%%%%%%%%%%%%%%%%%%%%%%%%%%%%%%%%%%%%%%%%%%%%%%%%%%%%%%%%%%%%%%%
% Ab hier bearbeiten 
%%%%%%%%%%%%%%%%%%%%%%%%%%%%%%%%%%%%%%%%%%%%%%%%%%%%%%%%%%%%%%%%%%%%%%%%%%%%%%%%%%%%%%%%%
Informationen zu \LaTeX{} auf z.B.: \url{https://tex.cloud.uni-hannover.de/learn}.

\section*{Aufgabe 1}
\begin{enumerate}[a)]
	\item $f \ll g \ll h$ % inline mathmode
	\begin{enumerate}[1.]
		\item $f_3(n) = \left(\sqrt{3n} + \sqrt{12n}\right) \cdot \left(-\sqrt{3n} + \sqrt{12n}\right) = -3n + 12n = 9n = O(n)$
		\item $f_5(n) = \frac{6n^2-18n-10}{(2n+1)(n-5)} = O(n^2)$

		\item $f_2(n) = 4^{1 + \log(n^2)} + 2^{4 \cdot \log(2^n)} = O(4^{\log(n^2)})$\\
			  $\lim_{n \to \infty}4^{log(n^2)} > (2^4)^{\log(2^n)}$
		
		\item $f_4(n) = \log(9n^2) + \log(2n^5) = O(\log(n^5))$
		\item $f_6(n) = \log\left(\sqrt{2 \pi n}\left(\frac{n}{e}\right)^{\!n}\right) = O(\log(n^n))$\\
		$\lim_{n \to \infty}\log(\sqrt{n}\ n^n)$\\
		$\lim_{n \to \infty}\log(n^{n+\frac{1}{2}})$\\
		$\lim_{n \to \infty}\log(n^n)$
		
		\item $f_1(n) = 42n + 17 + 120n^2 + 23n^3 = O(n^3)$

			  
	\end{enumerate}

	\item \textbf{Satz 1.22} Seien $f, g$ Funktionen. Falls $g \in O(f)$, dann folgt $f + g \in \Theta(f)$.
\end{enumerate}


\section*{Aufgabe 2}
\begin{enumerate}[a)]
	\item
	Graphen können mit dem Tikz-Paket erstellt werden\\
	\url{https://tex.cloud.uni-hannover.de/learn/latex/TikZ_package}\\
	\url{https://tobiw.de/tbdm/tikz-adventskalender}
	\begin{center}
	\begin{tikzpicture}
		\node[draw, circle] (0) {0};
		\node[draw, circle, right = of 0] (1) {1};
		\node[draw, circle, right = of 1] (2) {2};
		\node[draw, circle, below = of 0] (3) {3};
		\node[draw, circle, right = of 3] (4) {4};
		\node[draw, circle, right = of 4] (5) {5};
		\draw[-Stealth] (0) -- (1);
		\draw[-Stealth] (0) -- (3);
		\draw[-Stealth] (1) -- (5);
		\draw[-Stealth] (5) -- (2);
		\draw[-Stealth] (4) -- (1);
		\draw[-Stealth] (3) -- (1);
		\draw[-Stealth] (4) -- (0);
		\draw[-Stealth] (5) -- (4);
		\draw[-Stealth] (2) to[bend left] (5);
		\draw[-Stealth] (3) to[bend right] (5);
	\end{tikzpicture}
	\end{center}
	\item
	\item
\end{enumerate}


\section*{Aufgabe 3}
\begin{enumerate}[a)]
	\item \
	\begin{center}
	\begin{tikzpicture}[baseline=(0.center)]
		\node[draw, circle, fill=blue!33!white] (0) {0};
		\node[draw, circle, right = of 0] (1) {1};
		\node[draw, circle, right = of 1] (2) {2};
		\node[draw, circle, below = of 0] (3) {3};
		\node[draw, circle, right = of 3] (4) {4};
		\node[draw, circle, right = of 4] (5) {5};
		\draw[-] (0) -- (1);
		\draw[-] (0) -- (5);
		\draw[-] (1) -- (2);
		\draw[-] (1) -- (3);
		\draw[-] (2) -- (4);
		\draw[-] (3) -- (4);
		\draw[-] (3) to[bend right] (5);
	\end{tikzpicture}
	\qquad\qquad
	\begin{tikzpicture}[baseline=(0.center)]
		\node[draw, circle, fill=red!33!white] (0) {0};
		\node[draw, circle, right = of 0] (1) {1};
		\node[draw, circle, right = of 1] (2) {2};
		\node[draw, circle, below = of 0] (3) {3};
		\node[draw, circle, right = of 3] (4) {4};
		\node[draw, circle, right = of 4] (5) {5};
		\draw[-] (0) -- (3);
		\draw[-] (4) -- (1);
		\draw[-] (5) -- (4);
		\draw[-] (1) -- (2);
		\draw[-] (5) -- (2);
		\draw[-] (0) -- (1);
		\draw[-] (3) to[bend right] (5);
	\end{tikzpicture}
	\end{center}
	\item
	\item
\end{enumerate}
\end{document}